%%%%%%%%%%%%%%%%%%%%%%%%%%%%%%%%%%%%%%%%%%%%%%%%%%%%%%%%%%%%%%%%%%%%%%%%%%%%%%%%%%%%%%%%%%%%%%%%%%%%%%
%
%   Filename    : abstract.tex 
%
%   Description : This file will contain your abstract.
%                 
%%%%%%%%%%%%%%%%%%%%%%%%%%%%%%%%%%%%%%%%%%%%%%%%%%%%%%%%%%%%%%%%%%%%%%%%%%%%%%%%%%%%%%%%%%%%%%%%%%%%%%

\begin{abstract}
% From 150 to 200 words of short, direct and complete sentences, the abstract 
% should be informative enough to serve as a substitute for reading the thesis document 
% itself.  It states the rationale and the objectives of the research.  

% In the final thesis document (i.e., the document you'll submit for your final thesis defense), the 
% abstract should also contain a description of your research results, findings, 
% and contribution(s).

%OLD ABSTRACT

% This research explores the interaction among children learning through a playful mobile musical interface.  In using the innate playful nature of children, a sandbox environment will be implemented to supplement the experiences of learning music. We will design gestures that allows children to interact with a firefly model that represents different musical rudiments, such as rhythm, beat-rest patterns, notes, measures, and sections. This will be accomplished through the use of a human-centered design process of understanding children, and music teachers. The findings will guide the design and development of an iterative prototype that will repeatedly be tested by children. Continuous feedback through experiments and usability tests will allow us to discover what human factors are exhibited by children when doing music composition tasks.

 Music has many complex properties that make it difficult to learn especially for children. Various innovations have been introduced to make the music learning process exciting for them. However, much has yet to be understood on how children and their innate playfulness can be used as the best motivation in helping them learn difficult concepts in music. In this research, we first investigated how music experts teach children at an early age concepts such as tempo, rhythm, pitch and notes. Second, we rapidly developed a mobile musical application across several iterations and tested them with our participants. Third, we performed a user study to evaluate the application's usability, the children's learning performance, and music quality of their outputs. We found specific traits and activity patterns that are correlated with how children play and learn with music. These findings enabled us to uncover specific design affordances and guidelines that may effectively teach and engage children in the long term. 
 
% Music can trigger many types of memories in children and can be used to engage with them at any time. Yet music has many complex properties that make it difficult to learn. This research explores the playful interactions exhibited by children in learning through a playful mobile musical interface. In using the innate playful nature of children, a sandbox environment was implemented to supplement the experiences of learning music. We have designed gestures that allows children to playfully interact with a firefly model that represents different musical rudiments, such as tempo, rhythm, beat-rest patterns, pitch, notes, measures, and sections. This was accomplished through the use of a human-centered design process of understanding children, and music teachers. We have created a version 1 of the prototype and we have done usability testing on this version. The findings that were gathered have guided us in the design and development of the iterative prototype that was tested in one iteration by the children. Results so far have show that children after many repetitions of a task were keen in finding patterns in elements in the screen such as color and size that represented a specific element of music. For the upcoming second iteration we shall use the comments from the version 1 testing to come up with a version 2 prototype. More feedback through different iterations of these usability tests will allow us to discover more human factors that are exhibited by children when doing music composition tasks Lastly,the function of this tool will contribute in creating a child friendly interface for children to learn a hard concept like music. 


%
%  Do not put citations or quotes in the abstract.
%

% Keywords can be found at \url{http://www.acm.org/about/class/class/2012?pageIndex=0}.  Click the 
% link ``HTML'' in the paragraph that starts with ''The \textbf{full CCS classification tree}...''.

\begin{flushleft}
\begin{tabular}{lp{4.25in}}
\hspace{-0.5em}\textbf{Keywords:}\hspace{0.25em} & Human Computer Interaction, Music Representation, Sandbox Environment, Gestural Input, Usability testing, User Interface Design
\end{tabular}
\end{flushleft}
\end{abstract}

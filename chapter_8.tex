\chapter{Conclusion and Future Work}
\section{Summary}
In this study, we were able to understand how children were taught music concepts like tempo, rhythm, pitch, and notes by their music teachers. The findings can be seen in Appendix \ref{sec: preliminary}, where we were able to get their insights and converted them into artifacts that guided us in developing the application. We were then able to design and develop across several iterations the different features of the application as seen in Appendix \ref{sec:appendixf}, Appendix \ref{sec: it1design} and Section \ref{sec: it3results}. Also, through our user studies we were able to observe the applications usability through the children's outputs. The usability of the application was also evaluated using Attrakdiff as well as the scores given by the music expert evaluations based on the quality of the outputs given by the children. Lastly, we were able to discover the human factors that children exhibit when they are using our application for learning music. Categories of human factors include Behaviour, Attention, and Learning findings were observed as discussed in Section \ref{sec: humanFactors}.

The application was iteratively developed. Usability testing of the application was done along with using AttrakDiff. A high fidelity prototype was initially designed using the Figma. It consisted of images that depicted how the application would look like at different points. Users could also press on screen flows and hot spots that simulate the behaviour of FireflyX. We were then able to develop an initial prototype based from the mockup. Iterative testing commenced where users were asked to do specific tasks. This was done in order to examine the usability of the application. Based from the testing from the previous iterations, major changes such as changing the cookies into candies, removing the sun and moon, reducing the amount of patterns to five (5), were recommended. A third iteration testing using this version of the application focused on teaching children musical rudiments such as rhythm, tempo, and pitch. The outputs of the children were also graded by three music experts.

Based from the AttrakDiff results, the application shows that is has good usability and that it caters to the interests of the children. From the same tests, the overall attractiveness of the application seems to entertain the children users as well. Although words such as technical, cumbersome, and unpredictable appeared in the results, over all the application still performed relatively better than expected. 

\section{Recommendations}
% Future work
For future work, the authors believe that there is a need to perform these tests involving more participants and music experts. This will give us a broader understanding on the music learning of the children, the rating of experts and the human factors per se. We also recommend to test the application in a face-to-face environment as we believe more observations about the participants could be gathered than the virtual setup. This way, it will be easier for both the researchers and the participants to perform the tests. Since the current application only provides learning for an individual child, a collaborative feature can be introduced so that children can help their learning as a group by either configuring the fireflies together or sharing their firefly configurations while using the application.

Future work that relate to the scope of the application can explore expanding the notes, such as dotted notations, that the users can choose from. Other music rudiments such as harmony, timbre, and texture could be added to the application. These additions could broaden the scope of the application and might introduce curiosity to the users. Furthermore, by giving users more options and combinations to choose from, these could take advantage of the playfulness of children which would make them play more with the application.




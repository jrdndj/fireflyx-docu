\chapter{Human Factors in using FireflyX}

(Temporary for now will remove once correlations have been put and once the scores of music experts can be used as comparison)
(subsections are also temporary we just put it there for)

Through our user study, we performed analysis on both qualitative and quantitative data. These qualitative data include 

\subsection{Behavioral Findings}

We found out that children play along with the output of the firefly. P1, P2 and P3 demonstrated specific behaviours while trying to play and listen with the music. They begin by listening to the tune and nod or move their head along with the tune as they listen to verify the fireflies they have created. We then believe that this nodding behavior aids in memory as in a study by \citeA{de2018optimizing} also mentioned that a child nodding his head confirms a “yes” showing a positive kind of response.

In using the FireflyX application we found out that 4 out of 5 participants specifically P1, P2, P4, and P5 tap a lot and they also demonstrated that they were eager to explore the application and they want to try it out on their own. We believe that this behaviour is seen because the app implemented a sandbox environment and that the app encourages playful behaviour. This simulates that the children are able to emulate playing outside by using a sandbox environment in an application \cite{inal2007flow}. 

The participants exhibited happiness and being more energetic when they get their tasks right. Whenever they would hear their outputs and it would match the expected output they will be happy and they would also exhibit actions such as being pumped up and clapping in excitement. The children would also have an increase in mood when they would get their tasks right. Notably, P4 would also laugh and still be happy even if he gets something wrong. We believe that accomplishing tasks can give uplift a child's mood and give them a morale boost. A study by \citeA{elahi2017xylotism} also mentioned this in their application as when the child is validated and reassured that they are right it helps the child to be more encouraged in using the application. 

They also started to get more confident the more they used the application. Comments like "easy" and not hesitating to select finish when they are done with the tasks show they get used to the app the more time they use it. We believe that children get more confident in using an application the more time they invest in using it. As \citeA{frokjaer2000measuring} also stated that the time spent in using a system that shows that there is learning, they will eventually learn to do their tasks more efficiently.

We also observed that some participants are more annoyed and scared than others most notably P1 and P2 in making mistakes. They get frustrated and annoyed in making mistakes or when they don't get what is wrong with what they are doing which is observed more in doing tasks 4 and 5. This can be verified in the results of the Attrakdiff survey as for the word pair Undemanding-Challenging it leaned more toward challenging as tasks 4 and 5 were designed to challenge the learning of the children in using the application. More comments and observations of them being scarred of making mistakes was seen in these two harder tasks as comments like "is there a time limit" and "I am not sure" were said by most of the participants in performing the two tasks. We believe that children do not like committing mistakes, as in a study by \cite{hourcade2015child} mentioned the social aspect of this. It stated that as children copy how adults act and one of them is how they accomplish their tasks. While the children would only notice how the adults would successfully do their tasks, they would not notice the mistakes made along the way as they might not realize or know that was wrong. 

Connected to the previous behavior of not wanting to be wrong is the observation of them asking and looking at the guardian for validation of their work. The participants that had a guardian but most notably P3 and P4 which were the two youngest of the participants. As seen in Figure \ref{fig:allassistance} in the previous chapter it shows that the two were more reliant on their parents in getting the correct answer.

\subsection{Attention Findings}

We noticed that there was a difference between the participants in using the references and guides that we put in their tasks.
P1 and P4 were more reliant in the diagram with the candies compared to P2, P3, and P5. This was observed when P1 and P4 both reacted more negatively to task 4 when this diagram was replaced with a music sheet which the other 3 participants were fine with using. The 3 participants P2, P3, and P5 were also the ones with more knowledge in music compared to the other two. We also noticed that these 3 participants that used the music sheets more confidently and were exhibiting the thinking out loud protocol more. We believe that thinking out aloud aids in memorability as this is supported by a study made by \cite{gagne1962study}. They found that there is a memory advantage in thinking out loud or saying the words compared to keeping it to yourself in your head. 

An important finding that validates the design of our application is when we got comments about the visual elements that they liked. The children understood the concept of the fireflies and what we were surprised that all of the participants like the concept of the candies more. This can also be seen in the results of the Attrakdiff survey as the scores for Stylish, Creative, and Appealing word pairs were all 5 and above with one of them scoring 7 being the highest meaning they all liked the design of the application. The design of the application helped in attracting the children in using the application as studies by \cite{cohen2011young,burton2016music,chung2017designing} all explicitly mentioned the importance of having a child friendly design so that the children will be motivated in using the app. This also attracts their attention and will help them in having an increased pace in learning. 

\subsection{Learning Findings}

(Will do one per participant with the eval scores once the music evals have been completed)

We then noticed that the website that we made to guide them in doing the tasks was really helpful for them. All participants used the site in order to do majority of their tasks and relied on copying the site. Notably P2 and P3 were relying less on the site as they were more used to the application. (mention their higher score from expert once the evaluation forms is finalized). Having a guide can help them learn the application and the concepts faster. Having a guide can help them learn the application and the concepts faster \cite{pashler2007organizing}. Making the guides to be clear and to be arranged in an organizing way helps to stimulate learning as it would be easier to copy and then little by little get used in learning a particular topic. 

They learn to use the app and start memorizing the notes that corresponds to the candies. As mentioned earlier in the design the candies also helped in their learning. All participants also improved in remembering what the buttons do, and in turn they navigate through the application faster as the sessions go by. We believe that these children get used to the layout of an app the more they use it and they start to instinctively press the buttons that they need given the task \cite{wiedenbeck1999use},\cite{ibharim2014ibuat}. These studies mention that when the elements are designed well and are understandable by the child in a glance they will learn to use the app faster.


The participants also realized their mistakes as all participants have demonstrated that they knew that they had mistakes and differences with the correct answer. This was seen more in task 5 as every time they do trial and error in the beginning they know that the song they made is not yet same with the expected output. We believe that they are able to hear the differences in pitch and pattern using the sound outputs as they have an idea to distinguish the pitch and even more so if the pitch is really far. 



\subsection{Unique Findings of FireflyX}

Through our findings, all the participants were doing a trial and error learning by exploring the many different configurations that they can use in the app. They were constantly comparing the different pitches and pattern until they would get close to the output they wanted. This allowed the Children to see their own mistakes without being given the answer by the application. As we are using a sandbox environment, it is a place wherein the participants can explore all the possible outcomes until they find their answer.

\section{Correlations}

\subsection{Assistance vs Age}







